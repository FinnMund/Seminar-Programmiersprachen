\documentclass{beamer}
\usepackage{graphicx}
\usetheme{Madrid}        % Aussehen der Präsentation
\usecolortheme{default} % Farben
\usefonttheme{default}  % Schrift
\usepackage[backend=biber,style=ieee]{biblatex}
\addbibresource{sources.bib}

\title{A short Introduction to Zig}
\author{Finn Mund}

\begin{document}

\begin{frame}
  \titlepage
\end{frame}

\begin{frame}{Content}
  \tableofcontents
\end{frame}

\section{Introduction}
\section{History}
\section{Why use Zig?}
\section{What Niche is the language supposed to fill?}
\section{Prominent Projects}
\section{Infrastructure around the language}
\section{Code}
\section{Conclusion}


\begin{frame}{Introduction}
\begin{itemize}
  \item System programming languages: OS, compilers, embedded systems
  \item Direct hardware control and high performance
  \item Established languages: C (fast, unsafe), Rust (safe, complex)
  \item Zig: modern alternative between C and Rust
  \item Focus on simplicity, control, and tooling
\end{itemize}

\begin{figure}[h]
    \centering
    \hspace{2cm} % Verschiebt alles um 2cm nach rechts
    \begin{minipage}{0.6\textwidth}
        \includegraphics[width=\textwidth]{Unbenannt}
    \end{minipage}
    \hfill 
    \begin{minipage}{0.2\textwidth}
        \small \cite{engheim2022whatmakeszigunique}
    \end{minipage}
\end{figure}
\end{frame}


\begin{frame}{History}
\begin{itemize}
  \item First announced by Andrew Kelley in 2016
  \item 2018: Kelley works on Zig full-time
  \item 2020: Zig Software Foundation founded (non-profit)
  \item Still pre-1.0 (version 0.15.2 as of 2026)
\end{itemize}
\end{frame}


\begin{frame}{Why use Zig?}
\begin{itemize}
  \item C-like performance with full low-level control
  \item No garbage collection → predictable runtime behavior
  \item Explicit memory management via allocators
  \item Strong, explicit error handling (errors as values)
  \item Compile-time execution with \texttt{comptime}
  \item Built-in cross-compilation support
  \item Simple, readable syntax for system-level code
\end{itemize}
\end{frame}





\begin{frame}{What niche does Zig fill?}
\begin{itemize}
  \item Systems programming dominated by C and Rust
  \item C: fast and simple, but unsafe (leaks, overflows, null pointers)
  \item Rust: safe and fast, but complex and restrictive
  \item Zig builds on C, but improves safety explicitly
  \item Safety without garbage collection or hidden runtime
  \item Simpler model than Rust, fewer abstractions
  \item Positioned between C and Rust
\end{itemize}
\end{frame}

\begin{frame}{Bun}
\begin{itemize}
  \item High-performance JavaScript runtime
  \item All-in-one tool: runtime, package manager, test runner, bundler
  \item Drop-in replacement for Node.js
  \item Written in Zig, uses JavaScriptCore
  \item \textbf{Goal:} Faster startup, lower memory usage, simpler tooling
\end{itemize}
\end{frame}

\begin{frame}{TigerBeetle}
\begin{itemize}
  \item Distributed database for financial transactions (OLTP)
  \item Extremely high throughput and low, predictable latency
  \item Optimized for cost efficiency and long-term reliability
  \item Written in Zig for full control over memory and performance
  \item \textbf{Goal:} Reliable transaction processing at massive scale
\end{itemize}
\begin{figure}[h]
    \centering
    \hspace{2cm} % Verschiebt alles um 2cm nach rechts
    \begin{minipage}{0.6\textwidth}
        \includegraphics[width=\textwidth]{tiger}
    \end{minipage}
    \hfill 
    \begin{minipage}{0.2\textwidth}
        \small \cite{tigerbeetle2026}
    \end{minipage}
\end{figure}
\end{frame}

\begin{frame}{Mach Engine}
\begin{itemize}
  \item Game engine for high-performance, cross-platform development
  \item Minimal setup: Zig compiler as the only dependency
  \item Focus on modularity and developer experience
  \item Strong use of Zig’s cross-compilation features
  \item \textbf{Goal:} Simple, modern alternative to complex C++ engines
\end{itemize}
\end{frame}

\begin{frame}{Zig Language Server (ZLS)}
\begin{itemize}
  \item Language server for Zig (LSP-based)
  \item Provides autocompletion, diagnostics, and syntax highlighting
  \item Works with VS Code, Neovim, JetBrains IDEs
  \item Written in Zig itself
  \item \textbf{Goal:} Improve developer productivity and tooling support
\end{itemize}
\end{frame}

\begin{frame}{Code: Positive Aspects}
    \begin{itemize}
        \item \textbf{Standard Library:} Extensive (FS, Strings, ASCII).
        \item \textbf{Memory Control:} Explicit allocators \& \texttt{defer}.
        \item \textbf{Predictability:} Better resource management.
        \item \textbf{Error Handling:} \texttt{try} enforces robust code.
    \end{itemize}
\end{frame}

% Slide 2: Cons
\begin{frame}{Code: Challenges}
    \begin{itemize}
        \item \textbf{I/O Efficiency:} Hard-coded limits for large files.
        \item \textbf{Rapid Evolution:} Frequent breaking changes.
        \item \textbf{Outdated Docs:} Community info often outdated.
        \item \textbf{Library Gaps:} Limited function documentation.
    \end{itemize}
\end{frame}

\begin{frame}{Conclusion}
\begin{itemize}
  \item Zig combines proven low-level concepts with modern approaches to memory management
  \item Explicit control enables safety improvements without sacrificing performance
  \item Real-world projects demonstrate Zig’s viability despite its pre-1.0 status
  \item The non-profit foundation model builds trust and supports long-term innovation
  \item Rapid language evolution remains a challenge, especially for new developers
\end{itemize}
\end{frame}

\begin{frame}[allowframebreaks]{Sources -- URLs}
\scriptsize
\begin{itemize}
  \item https://andrewkelley.me/post/intro-to-zig.html
  \item https://andrewkelley.me/
  \item https://andrewkelley.me/post/full-time-zig.html
  \item https://ziglang.org/news/announcing-zig-software-foundation/
  \item https://ziglang.org/zsf/
  \item https://tigerbeetle.com/blog/2025-10-25-synadia-and-tigerbeetle-pledge-512k-to-the-zig-software-foundation/
  \item https://ziglang.org/news/2025-financials/
  \item https://ziglang.org/download/
  \item https://www.tiobe.com/tiobe-index/
  \item https://www.geeksforgeeks.org/c/security-issues-in-c-language/
  \item https://www.netdata.cloud/academy/how-to-find-memory-leak-in-c/
  \item https://www.code-intelligence.com/blog/buffer-overflows-complete-guide
  \item https://learn.snyk.io/lesson/null-dereference/?ecosystem=cpp
  \item https://zig.guide/language-basics/defer/
  \item https://ziglang.org/documentation/master/std/#std.testing
  \item https://zig.guide/language-basics/slices/
  \item https://zig.guide/language-basics/pointers/
  \item https://zig-by-example.com/pointers
  \item https://dev.to/mukhilpadmanabhan/rust-vs-zig-the-new-programming-language-battle-for-performance-1p6
  \item https://bun.com/docs
  \item https://docs.tigerbeetle.com/start/
  \item https://github.com/zigtools/zls
  \item https://rock-the-prototype.com/programmieren-lernen/node-js/
  \item https://docs.tigerbeetle.com/concepts/performance/
  \item https://ziglang.org/learn/tools/
  \item https://ziglang.org/documentation/master/#Build-Dependencies
  \item https://ziglang.org/learn/build-system/
  \item https://ziggit.dev/
  \item https://ziglang.org/documentation/master/std/
  \item https://ziglang.org/documentation/master/#Zig-Test
  \item https://www.invensis.net/blog/applications-of-c-and-c-plus-plus-in-the-real-world
  \item https://en.wikipedia.org/wiki/System_programming_language
  \item https://ziglang.org/learn/overview/
  \item https://dev.to/hexshift/manual-memory-management-in-zig-allocators-demystified-46ne
  \item https://zig.guide/language-basics/errors/
  \item https://ziglang.org/documentation/master/#Errors
  \item https://zig.guide/language-basics/comptime/
  \item https://zig.guide/build-system/cross-compilation/
  \item https://ziglang.org/learn/why_zig_rust_d_cpp/
  \item https://machengine.org/about/goals/
  \item https://machengine.org/
\end{itemize}
\end{frame}
\printbibliography


\end{document}
